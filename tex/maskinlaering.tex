\documentclass[norsk]{beamer}				% frames
%\documentclass[notes, norsk]{beamer}		% frames + notes
%\documentclass[notes=only]{beamer}	% notes
\input{preamble.tex}
\input{frames.tex}
\input{defs.tex}

%%%%%%%%%%%%%%%%%%%%%%%%%%%%%%%%%%%%%%%%%%%%%%%%%%%%%%%%%%%%%%%%%%%%%%%%%%%%%%%%%%%%%%%%%%%%%%%%%%%%%%%%%%%%%%%%%%%%%
% SPECIFY INFORMATION TO BE USED IN FRONT FRAME
%%%%%%%%%%%%%%%%%%%%%%%%%%%%%%%%%%%%%%%%%%%%%%%%%%%%%%%%%%%%%%%%%%%%%%%%%%%%%%%%%%%%%%%%%%%%%%%%%%%%%%%%%%%%%%%%%%%%%

\newcommand{\mtitle}{Maskinlæring}
\newcommand{\mauthor}{Even Marius Nordhagen}
\newcommand{\mmail}{evenmn@fys.uio.no}
\newcommand{\massignn}{.}

\begin{document}

\frontframe

\note{
	\begin{itemize}
		\item This is an example presentation about quantum mechanics
		\item The front frame is generated using \textit{frontframe}
		\item Note also that the notes can be turned on and off in the first line of this file
	\end{itemize}
}

\mframe{Oversikt}{}{
	\begin{itemize}
		\setlength\itemsep{1em}	% This line specifies the spacing between bullet points
		\item Motivasjon
		\item Teorien bak
		\item Implementasjon
		\item Dere skal implementere et nevralt nettverk
	\end{itemize}
}

\note{Dette er planen for dagen}

\titleframe{Motivasjon}

\mframe{Regresjon}{}{
	Tilpasse kurve til et sett med punkter
	
	Enkel form for regresjon
}

\note{Konsept dere kanskje er kjente med}

\mframe{Stemmegjenkjenning}{}{
	Kjenne igjen stemmer
}

\note{Kjenne igjen stemmer}

\mframe{Bildeanalyse}{}{
	Kjenne igjen hva som er på et bilde
	
	Kjenne igjen hvor på bildet vi finner et objekt
	
	Bedre enn mennekser
}

\notes{kfkffk}

\frame{Generative modeller}{}{
	
}

\mframe{}{}{
	\begin{shadequote}{
			The general theory of quantum mechanics is now almost complete... ...The underlying physical laws necessary for the mathematical theory of a large part of physics and the whole of chemistry are thus completely known, and the difficulty is only that the exact application of these laws leads to equations much too complicated to be soluble. \par Paul M. Dirac, \emph{Quantum Mechanics of Many-electron Systems} \supercite{dirac_paul_adrien_maurice_quantum_1929}}
	\end{shadequote}
}

\note{Frame without title or subtitle}

\mframe{The Schrödinger Equation}{The time-independent Schrödinger equation}{
	The time-independent Schröinger equation is given by
	\begin{empheq}[box={\mybluebox[5pt]}]{equation}
		\hat{\mathcal{H}}\Psi_n=\varepsilon_n\Psi_n,
	\end{empheq}
	with $\hat{\mathcal{H}}$ as the Hamiltonian, $\Psi_n$ as the wave function and $\varepsilon_n$ as the corresponding energy \supercite{schrodinger_undulatory_1926}. 
}
\note{Here, we have a basic slide with subtitle}

\mframe{The Probability Distribution}{}{
	The probability distribution in quantum mechanics is given by
	\begin{empheq}[box={\mybluebox[5pt]}]{equation}
		P(\boldsymbol{r})=\frac{\Psi_n(\boldsymbol{r})^*\Psi_n(\boldsymbol{r})}{\int d\boldsymbol{r}\Psi_n(\boldsymbol{r})^*\Psi_n(\boldsymbol{r})},
	\end{empheq}
	where $\boldsymbol{r}$ is a set of spatial and spin coordinates \supercite{born_zur_1926}.
	\pause
	However, often the wave function is assumed to be normalized, and the equation is simply written as
	\begin{equation}
	P(\boldsymbol{r})=\Psi_n(\boldsymbol{r})^*\Psi_n(\boldsymbol{r}).
	\end{equation}
	\note<1>{Here is a basic slide without a subtitle.}
	\note<2>{The pause function can be used to add more elements to a slide}
}

\titleframe{Thank you!}

\note{The title frame contains just a large centered text (should not be confused with frontframe)}

\mframe{References}{}{
	\printbibliography
}

%\appendix

% Backup slides

%\titleframe{Some Backup Slides}

%\mframe{Backup slides}{}{
%	\begin{figure}
%	\centering
%	\input{../tikz/biasvariancedecomp.tex}
%	\end{figure}
%}


\end{document}